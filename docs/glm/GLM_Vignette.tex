\documentclass[11pt]{article}
\usepackage[english]{babel}
\usepackage{amsmath,amsthm,amsfonts,amssymb,epsfig}
\usepackage[left=1.1in,top=1in,right=1.1in]{geometry}
\usepackage{url}
\usepackage{array}
\usepackage{datetime}
\usepackage{lipsum}
\usepackage{url}
\usepackage{spverbatim}
\usepackage{hyperref}
\hypersetup{colorlinks, urlcolor={blue}}
\usepackage{graphicx}
\graphicspath{ {images/} }

\begin{document}

\thispagestyle{empty} %removes page number

\begin{center}
\textsc{\Large\bf{Generalized Linear Modeling  with H2O's R Package}}
\\
\bigskip
\textsc{November 2014}
\end{center}
\bigskip
\bigskip
\bigskip
\bigskip
\tableofcontents

\newpage


\section{Introduction} \label{1}

Generalized linear models (GLM) are some of the most commonly-used models for many types of data analysis use cases. While some data analysis can be done using general linear models, if the variables are more complex, general linear models may not be as accurate. For example, if the dependent variable has a non-continuous distribution or if the effect of the predictors is not linear, generalized linear models will produce more accurate results than general linear models.  

H2O's GLM algorithm fits the generalized linear model with elastic net penalties. The model fitting computation is distributed, extremely fast,and scales extremely well for models with a limited number (~ low thousands) of predictors with non-zero coefficients . The algorithm can compute models for a single value of a penalty argument or the full regularization path, similar to glmnet. It can compute gaussian (linear), logistic, poisson, and gamma regression models. This document describes the usage of H2O's implementation of GLM in R.

H2O's GLM fits the model by solving following problem:

\[ \min\limits_{\beta}\ {{1\over{N}} log-likelihood(family,\beta)  + \lambda (\alpha \| \beta \|_1}  + {1- \alpha \over 2} \| \beta \|_2) \]

The elastic net parameter $\alpha$ controls the penalty distribution between L1 and L2 penalty. It can have any value between 0 and 1. When $\alpha$ = 0, we have no L1 penalty and the problem becomes ridge regression. If $\alpha$ = 1, there is no L2 penalty and we have lasso.

The main advantage of an L1 penalty is that with sufficiently high $\lambda$, it produces a sparse solution; the L2-only penalty does not reduce coefficients to exactly 0. The two penalties also differ in the case of correlated predictors. The L2 penalty shrinks coefficients for correlated columns towards each other, while the L1 penalty will pick one and drive the others to zero. Using the elastic net argument $\alpha$, you can combine these two behaviors. It is also useful to always add a small L2 penalty to increase numerical stability.

As mentioned before, model-fitting works and scales well if numcols $<<$ numrows. However, in many cases we want to fit a GLM model over many predictors (10,000 or more) with an L1 penalty, selecting only a few of them for the model. The rest will have a zero coefficient and will not be part of the model. To maintain excellent performance and scaling properties in such cases, we employ strong rules  (as described in "Strong Rules for Discarding Predictors in Lasso-type Problems") to limit the set of active coefficients. The performance of the algorithm is then limited by the number of coefficients in the model, rather than the number of coefficients present in the training data set. 

The algorithm can handle categorical variables, which are treated as a special kind of sparse vectors: each categorical column is expanded on the fly into multiple binary vectors. 



To prevent such a correlation, we usually skip one of the levels (the first one). This is not necessary with a non-zero penalty and H2O has an option to include or factor levels in the model. All levels are always included when computing the whole regularization path. This is useful when interpreting the variable importance of the model. 

\section{Installation} 

You can load the latest CRAN H2O package by running:

\begin{spverbatim}
install.packages("h2o")
\end{spverbatim}
\bigskip
\noindent
Alternatively, you can (and should for this tutorial) download the latest H2O build by following the ``Install in R" instructions in the H2O download table:\underbar{http://s3.amazonaws.com/h2o-release/h2o/master/latest.html}. Open your R Console and run the following to install the latest H2O build in R:

\begin{spverbatim}
# The following two commands remove any previously installed H2O packages for R.
if ("package:h2o" %in% search()) { detach("package:h2o", unload=TRUE) }

if ("h2o" %in% rownames(installed.packages())) { remove.packages("h2o") }

# Next, we download, install and initialize the H2O package for R (filling in the *'s with the matching digits in the download table)
install.packages("h2o", repos=(c("http://s3.amazonaws.com/h2o-release/h2o/master/
****/R", getOption("repos"))))

library(h2o)

\end{spverbatim}
\noindent
If you are operating on a single node, initialize H2O with

\begin{spverbatim}
h2o_server = h2o.init()

\end{spverbatim}
\noindent
With this command, the H2O R module will start an instance of H2O automatically at localhost:54321. Alternatively,  to specify a connection with an existing H2O cluster node (other than localhost at port 54321) you must explicitly state the IP address and port number in the \texttt{h2o.init()} call. An example is given below, but you should specify the IP and port number appropriate to your specific environment.

\begin{spverbatim}
h2o_cluster = h2o.init(ip = "192.555.1.123", port = 12345, startH2O = FALSE)

\end{spverbatim}
\noindent
An automatic demo is available to see \texttt{h2o.glm} at work. Run the following command to observe an example classification model built through H2O's GLM:

\begin{spverbatim}
demo(h2o.glm)
\end{spverbatim}

\subsection{Support} 

Users of the H2O package may submit general enquiries and bug reports to the 0xdata support address:\underbar {h2ostream@googlegroups.com}. Alternatively, specific bugs or issues may be filed to the 0xdata JIRA: \underbar{https://0xdata.atlassian.net/secure/Dashboard.jspa}.

\section{Generalized Linear Modeling} 
This section contains a brief overview of generalized linear models and follows up with a few details for each model family.

The generalized linear model is a generalization of linear regression. Linear regression models the dependency of response y on a vector of predictors x ($y \sim x^T \beta + \beta_0$), with the assumption that y has a gaussian distribution, with the mean being a linear function of x (+ the offset) and a deviance of $\sigma$ (independent of x), i.e. $ y = \mathcal{N}(x^T \beta + \beta_0 , \sigma^2) $. In many situations, these assumptions are overly restrictive but we would still want to take advantage of linear models. GLM generalizes linear regression in following ways: 
\begin{itemize} 
\item adds a non-linear link function, so that $link(y) \sim x^T \beta + \beta_0$.
\item allows variance to depend on the predicted value (family argument).

\end{itemize}
This generalization allows us to apply GLM to problems such as binary classification (Logistic regression).
GLM models are fitted by maximizing the likelihood.

\subsubsection{Model Fitting}
GLM models are fitted by maximizing the likelihood. For gaussian family, maximum likelihood is simply minimal mean squared error which has an analytical solution and it can be solved with ordinary least squares. For all other families, the maximum likelihood problem has no analytical solution and we have to use an iterative method (typical examples are IRLSM, Newton method, gradient descent, L-BFGS). 
\subsubsection{Model Validation}
Evaluating quality of the model is critical part of any data-modelling/prediction task. There are several standard ways how to evalute quality of fitted GLM model. The most common is to judge by deviance statistic. Deviance is based on comparing log likelihood of the fitted model with log likelihood of the saturated model (theoretical perfect model).

\[ deviance = 2({ln(L_{s})} - {ln(L_{m})}) \]

Another metric frequently used for model selection is The Akaike information criterion (AIC). AIC is a measure of the relative quality of a statistical model for a given set of data. It is usefull for model comparison. Unlike deviance (which would have perfect value for saturated model), it takes into account the complexity of the given model. It is based on meassuring information loss when replacing original data with the model itself. Unlike deviance, it does not meassure absolute quality of the fit (comparison against null-hypothesis). AIC is defined as follows:

\[ aic = 2k - 2ln(L_{m})\]
, where k is number of model parameters and L is likelihood of the model over the dataset.

\subsection{Regularization with Elastic Net Penalty} 
We introduce penalty to model-fitting to avoid over-fitting, reduce variance of the prediction error and to deal with correlated predictors. There are two common penalized linear models, Ridge Regression and Lasso. Ridge regression provides greater numerical stability and is easier (faster) to compute. On the other hand, Lasso leads to a sparse solution which is big advantage in many situations as it can be used for feature selection and produce models with fewer parameters. When ecnoutering highly correlated columns, l2 penalty tends to push all of the coefficients towards each other, while l1 penalty will pick one and remove the others (0 coefficients).

Elastic net combines the two and adds another parameter $\alpha$ which controls distribution of the pnelaty between l1 and l2. The combination of the two penalties is benefitial since l1 gives sparsity while l2 gives stability and encourages grouping effect (group of correlated variables tends to be dropped or aded into the model all at once).One possible use of the $\alpha$ argument is to do lasso with very little l2 penalty ($\alpha$ almost 1) to stabilize the computation (improve convergence speed)

Model fitting problem with elastic net penalty becomes:

\[ \min\limits_{\beta,\beta_0}\ {{1\over{N}} ln(L(family,\beta,\beta_0))  + \lambda (\alpha \| \beta \|_1}  + {1- \alpha \over 2} \| \beta \|_2^2) \]


\subsection{GLM Models} 
The following subsection describes GLM families supported in h2o. 

\subsubsection{Linear Regression (Gaussian family) }
Linear regression refers to the gaussian family model. It is the simplest example of GLM, but it has many uses and several advantages over the other families, such as faster and more stable computation. 

It models the dependency as a purely linear function (with link = identity):
\[ \hat{y} = x^T\beta + \beta_0\]

The model is fitted by solving the least squares problem (maximum likelihood for gaussian family):

\[ \min\limits_{\beta,\beta_0} { {1 \over 2N}\sum\limits_{i=1}\limits^{N}(x_i^{T}\beta  + \beta_0- y_i)^T (x_i^{T}\beta + \beta_0 - y_i))  + \lambda (\alpha \|\beta \|_1 + {1-\alpha \over 2}) \| \beta \|_2^2} \]


Deviance is simply the sum of squared errors:
\[ D = \sum\limits_{i=1}\limits^{N}{(y_i - \hat{y}_i)^2} \]


\textbf{Example}



\subsubsection{Logistic Regression (Binomial Family)}
Logistic regression can be used in case of a binary classification problem, where the response is categorical with two levels. It models dependency as $Pr(y = 1|x)$. The canonical link for binomial family is logit (log of the odds), its inverse is a logistic function that takes any real number on the input and projects it onto the 0,1 range (s-curve). 

\[ \hat{y} = Pr(y = 1|x) = {e^{x^T\beta + \beta_0} \over 1 + e^{x^T \beta + \beta_0} } \]

or alternatively:


\[log {\hat{y} \over 1- \hat{y}} = log{Pr(y=1|x) \over Pr(y=0|x)} = x^T \beta + \beta_0\]

The model is fitted by solving:
\[  \min\limits_{\beta,\beta_0} { {1 \over N}\sum\limits_{i=1}\limits^{N}(y_i (x_i^{T}\beta  + \beta_0) - log (1 + e^{x_i^{T}\beta  + \beta_0} )  + \lambda (\alpha \|\beta \|_1 + {1-\alpha \over 2}) \| \beta \|_2^2} \]

Deviance is -2 log likelihood:
\[D = -2\sum\limits_{i=1}\limits^{N}{(y log(\hat{y}) + (1 - y)log(1-\hat{y})  )}\]

\textbf{Decision threshold}

\textbf{Example}

\subsubsection{Poisson Models}
Poisson regression is generally used in cases where the response represents counts and we assume errors have a Poisson distribution. In general, it can be applied to any data where the response is non-negative. 

When building a Poisson model, we usually model dependency of the mean on the log scale, i.e. canonical link is log and prediction is:

\[\hat{y} = e^{x^T\beta + \beta_0}\]

The model is fitted by solving:

\[  \min\limits_{\beta,\beta_0} { {1 \over N}\sum\limits_{i=1}\limits^{N}(y_i (x_i^{T}\beta  + \beta_0) - e^{x_i^{T}\beta  + \beta_0})  + \lambda (\alpha \|\beta \|_1 + {1-\alpha \over 2}) \| \beta \|_2^2} \]

Deviance is 

\[D = -2\sum\limits_{i=1}\limits^{N}{(y log(\hat{y}) - y - \hat{y}}\]


\textbf{Example}

\subsubsection{Gamma Models}
The gamma distribution is useful for modeling a positive continuous response variable, where the conditional variance of the response grows with its mean but  the coefficient of variation of the response $\sigma^2(x)/μ(x)$ is constant for all x,  i.e., it has a constant coefficient of variation.

It is usually used with inverse or log link, inverse is the canonical link.

\textbf{Example}
% Strong Rules paper


\section{GLM on H2O}
This sections describes specifics of GLM implementation on h2o such as selecting regularization parameters and handling of categoricals. 

H2O's glm implementation presents a high performacne distributed algorithm which scales linearly with the number of rows and works extremely well for datasets with limited number of active predictors. 

\subsection{Summary of features} 
In summary, H2O's GLM functionalities include:

\subsection{Summary of features} 
In summary, H2O's GLM functionalities include:


\begin{itemize} 
\item fits generalized linear model with elastic net penalty
\item supported GLM families include Gaussian, Binomial, Poisson and Gamma
\item efficient handling of categorical variables
\item efficient computation full regularization path
\item efficient distributed n-fold cross validation
\item distributed grid search over elastic-net parameter $\alpha$
\item upper and lower boudns for coefficients
\item proximal operator interface
\end{itemize}

\subsection{Performance Characteistics}
The implementation is based on iterative reweighted least squares with ADMM inner solver to deal with L1 penalty. Every iteration of the algorithm consists of following steps:
\begin{enumerate} 
\item generate weigthed least squares problem based on previous solution, i.e. vector of weights w and reponse z 
\item compute the weighted gram matrix $X^TWX$ and $X^Tz$ vector
\item decompose the gram matrix (Cholesky decomposition) and apply ADMM solver to solve l1 penalized least squares problem
\end{enumerate}

Steps 1 and 2 are performed distributed, step 3 is computed in parallel on a single node. We can thus characterize the computational complexity and scalability of dataset with M observations and N columns (predictors) on a cluster with n nodes with p cpus each as follows:

\[ O({MN^2 \over pn} + {N^3 \over p})\]

And the overall memory cost is given by:

\[ O(MN + N^2pn)\]

In case of $M >> N$ we can forget the second term and the algorithm scales linearly both in the number of nodes and number of cpus per node. However, the equation above also implies that our algorithm is limited in the number of predictors it can handle since the size of the Gram matrix grows quadratically (memory and network throughput issue) with the number of predictors and it's decomposition cost grows as cube of the number of predictors (computational cost issue). H2O can get around these limitation in many cases due to its handling of categoricals and employing strong rules to filter out inactive predictors, both are described later in this chapter.  

\subsection{Categorical Variables}
When applying linear models to datasets with categorical varibales, typical approach is to expand the categoricals into set of binary vectors, one vector per each categorical level (e.g. by calling model.matrix in R). H2O performs similar expansion automaticaly so no changes to the dataset are needed. Each categorical column is then treated as a set of sparse binary vectors, with exactly one of them being 1 per line. Since such expansion leads to correlation with intercept column, another typical 
\subsection{Cross-Validation}
All validation values can be computed either on training data set (the default) or using nfold crossvalidation ($nfolds > 1$). When using nfold cross-validation, we randomly split data into n equally sized parts and train each of the n models on n-1 parts and compute validation on the part which was not used for training.The reported validation parameters are then obtained as follows:
\begin{itemize} 
\item null deviance is sum of null deviances of n-models (each uses null model based on the subset of the data)
\item residual deviance is sum of residual deviances of all n-models.
\item AIC
\item AUC is based on ROC curve build  by summing up confusion matrices built for all n-models.
This means for each threshold, we get confusion matrix which includes all the rows from
the training set, however, each row is classified exclusively by the model
which did not have it in its training set. Computation of AUC itself is then the same as in non-crossvalidatied case. 
\end{itemize}

\subsection{Selecting Penalty Parameters}

Since the L1 penalty produces a sparse solution, penalized GLM with  $\alpha > 0$ where a sufficiently high value of $\lambda$ will produce a null model (i.e., a model with no non-zero coefficients except the intercept). As a result, it only makes sense to compute the models for $\lambda < \lambda_{max}$, where $lambda_{max}$ is the smallest $\lambda$ leading to a null solution.

\subsubsection{Grid Search}
\subsubsection{Lambda Search}
Lambda search is essentially an automatic efficient grid search over the $\lambda$ parameter. It computes GLM models for the full regularization path, which is an exponentially decreasing sequence of values, starting at $\lambda_{max}$ and finishing with $\lambda_{min} = \gamma\lambda_{max}, \gamma \in (0,1)$ . $\lambda_{max}$ is defined as the smallest $\lambda$ s.t. the resulting model has no non-zero coefficients other than intercept. Since 



\subsection{Strong Rules}

\subsection{Handling Datasets High Number of Predictors}
\subsection{Coefficient Constraints}
upper,lower bounds, proximal operator interface

\section{Use case: Classification with Airline data}


\subsection{Airline dataset overview} 

The Airline dataset can be downloaded here: \url{https://github.com/0xdata/h2o/blob/master/smalldata/airlines/allyears2k_headers.zip}. Remember to save the .csv file to your working directory by clicking  "View Raw."  Before running the Airline demo, we'll first review how to load data with H2O. 

\subsubsection{Loading data} \label{2.5}

Loading a dataset in R for use with H2O is slightly different from the usual methodology, because we must convert our datasets into \texttt{H2OParsedData} objects. In this example, we will use a toy weather dataset that can be downloaded here: \underbar{https://raw.githubusercontent.com/0xdata/h2o/master/smalldata/weather.csv}. First, load the data to your current working directory in your R Console (do this for any future dataset downloads), and then run the following command:
\begin{spverbatim}
weather.hex = h2o.uploadFile(h2o_server, path = "weather.csv", header = TRUE, sep = ",", key = "weather.hex")
\end{spverbatim}
\bigskip
\noindent
To see a quick summary of the data, run the following command:
\begin{spverbatim}
summary(weather.hex)
\end{spverbatim}


\subsection{Performing a trial run} \label{3.2}
Returning to the Airline dataset demo, we first load the dataset with H2O and select the variables we want to use to predict a chosen response. For example, we can model if flights are delayed based on the departure's scheduled day of the week and day of the month.
\begin{spverbatim}

#Load the data and prepare for modeling
% @TODO h2o_server not found FIXXXXXXX & CHECK VARIABLES
air_train.hex = h2o.uploadFile(h2o_server, path = "AirlinesTrain.csv", header = TRUE, sep = ",", key = "airline_train.hex")

air_test.hex = h2o.uploadFile(h2o_server, path = "AirlinesTest.csv", header = TRUE, sep = ",", key = "airline_test.hex")

x = c("Year", "Month", "DayofMonth", "DayOfWeek", "UniqueCarrier", "FlightNum", "Origin", "Dest", "Distance")
y = "IsArrDelayed" 

\end{spverbatim}

Now we train the GLM model:

\begin{spverbatim}
airline.glm <- h2o.glm(x=x, 
                     y=y, 
                     data=airline.train.hex,
                     key = "glm_model",
                     family="binomial",
                     lambda_search = TRUE,
                     return_all_lambda = TRUE,
                     use_all_factor_levels = TRUE,
                     variable_importances = TRUE)
\end{spverbatim}

\subsubsection{Extracting and handling the results} \label{3.2.1}

We can extract the parameters of our model, examine the scoring process, and make predictions on new data.

%@ NOTE typoe in R demo test… i spelled performance wrong :(
\begin{spverbatim}
print("Check performance and AUC")
perf = h2o.performance(air.results$YES,airline.test.hex$IsArrDelayed )
print(perf)
perf@model\$auc
print("Show distribution of predictions with quantile.")
quantile.H2OParsedData(air.results\$YES)  
print("Extract strongest predictions.")
top.air <- h2o.assign(air.results[air.results\$YES > quant[‘75%] ],key="top.air")
top.air
\end{spverbatim}
\noindent
\\
\\
Once we have a satisfactory model, the \texttt{h2o.predict()} command can be used to compute and store predictions on the new data, which can then be used for further tasks in the interactive modeling process.
\begin{spverbatim}
#Perform classification on the held out data
prediction = h2o.predict(airline.glm, newdata=air_test.hex)
#Copy predictions from H2O to R
pred = as.data.frame(prediction)
head(pred)
\end{spverbatim}
\subsection{Web interface} \label{3.3}
H2O R users have access to an intuitive web interface to mirror the model building process in R. After loading data or training a model in R, point your browser to your IP address and port number (e.g., localhost:54321) to launch the web interface. From here, you can click on \textsc{Admin} $>$ \textsc{Jobs} to view your specific model details. You can also click on \textsc{Data} $>$ \textsc{View All} to track datasets currently in use. 
\subsubsection{Variable importances} \label{3.3.1}
One useful feature is the variable importances option, which can be enabled with the additional argument \texttt{importance=TRUE}. This feature allows us to view the absolute and relative predictive strength of each feature in the prediction task. From R, you can access these strengths with the command \texttt{air.model@model\$varimp}. You can also view a visualization of the variable
importances on the web interface.
\subsubsection{Java model} 
Java models are currently not available for GLM.
\\
\\
To download the model, open the terminal window, create a directory where the model will be saved, set the new directory as the working directory, and follow the curl and java compile commands displayed in the instructions at the top of the Java model.

\newpage
\section{Appendix A: Complete parameter list}
\begin{itemize}
\item \texttt{x}: A vector containing the names of the predictors in the model. No default.
\item \texttt{y}: The name of the response variable in the model. No default.
\item \texttt{data}: An \texttt{H2OParsedData} object containing the training data. No default.
\item \texttt{key}: The unique hex key assigned to the resulting model. If none is given, a key will automatically be generated.
\item \texttt{family}: A description of the error distribution and corresponding link function to be used in the model. Currently, Gaussian, binomial, Poisson, gamma, and Tweedie are supported. When a model is specified as Tweedie, users must also specify the appropriate Tweedie power. No default.
\item \texttt{link}: The link function relates the linear predictor to the distribution function. Default is the canonical link for the specified family. The full list of supported links: 
	gaussian: identity, log, inverse 
	binomial: logit, log 
	poisson: log, identity
	gamma: inverse, log, identity
	tweedie: tweedie 
\item \texttt{nfolds}: A logical value indicating whether the algorithm should conduct classification. Otherwise, regression is performed on a numeric response variable.
\item \texttt{nfolds}: Number of folds for cross-validation. Default is 0.
\item \texttt{validation}: An \texttt{H2OParsedData} object indicating the validation dataset used to construct confusion matrix. If  blank, the default is the training data.
\item \texttt{alpha}: The elastic-net mixing parameter, which must be in $[0,1]$. The penalty is defined to be $P(\alpha,\beta) = (1-\alpha)/2||\beta||_2^2 + \alpha||\beta||_1 = \sum_j [(1-\alpha)/2 \beta_j^2 + \alpha|\beta_j|] $ so \texttt{alpha=1} is the lasso penalty, while \texttt{alpha=0} is the ridge penalty. Default is 0.5.
\item \texttt{nlambda}: The number of lambda values when performing a search. Default is -1.
\item \texttt{lambda.min.ratio}: Smallest value for lambda as a fraction of lambda.max, the entry value, which is the smallest value for which all coefficients in the model are zero. Default is -1.
\item \texttt{lambda}: The shrinkage parameter, which multiplies $P(\alpha,\beta)$ in the objective. The larger lambda is, the more the coefficients are shrunk toward zero (and each other). Default is 1e-5.
\item \texttt{epsilon}: Number indicating the cutoff for determining if a coefficient is zero. Default is 1e-4.
\item \texttt{standardize}: Logical value indicating whether the data should be standardized (set to mean = 0, variance = 1) before running GLM. Default is true.
\item \texttt{prior}: Prior probability of class 1. Only used if \texttt{family = "binomial"}. Default is the frequency of class 1 in the response column. 
\item \texttt{variable\_importances}: A logical value (either TRUE or FALSE) to indicate whether the variable importances should be computed.  Compute variable importances for input features. NOTE: If \texttt{use\_all\_factor\_levels} is disabled,  the importance of the base level will NOT be shown. Default is false.
\item \texttt{use\_all\_factor\_levels}: A logical value (either TRUE or FALSE) to indicate whether all factor levels should be used. By default, the first factor level is skipped from the possible set of predictors. Set this flag if you want use all of the levels. Note: Needs sufficient regularization to solve! Default is false.
\item \texttt{tweedie.p}: The index of the power variance function for the tweedie distribution. Only used if \texttt{family = “"tweedie"}. Default is 1.5.
\item \texttt{iter.max}: Maximum number of iterations allowed. Default is 100.
\item \texttt{higher\_accuracy}: A logical value indicating whether to use line search. This will cause the algorithm to run slower, so generally, it should only be set to TRUE if GLM does not converge otherwise. Default is false.
\item \texttt{lambda\_search}: A logical value indicating whether to conduct a search over the space of lambda values, starting from \texttt{lambda\_max}. When this is set to TRUE, lambda will be interpreted as \texttt{lambda\_min}. Default is false.
\item \texttt{return\_all\_lambda}: A logical value indicating whether to return every model built during the lambda search. Only used if \texttt{lambda\_search = TRUE}. If \texttt{return\_all\_lambda = FALSE}, then only the model corresponding to the optimal lambda will be returned. Default is false.
\item \texttt{max\_predictors}: When \texttt{lambda\_search = TRUE}, the algorithm will stop training if the number of predictors exceeds this value. Ignored when \texttt{ lambda\_search = FALSE} or \texttt{max\_predictors = -1}. Default is -1.
\item \texttt{offset}: Column to be used as an offset, if you have one. No default.
\item \texttt{has\_intercept}: A logical value indicating whether or not to include the intercept term. If there are factor columns in your model, then the intercept must be included. Default is true.

%Additional: 
%Strong Rules Enables: Uses strong rules to filter out inactive columns. Default is true.
%Ignored_cols: No default.
%Non-negative: Restricts coefficients to be non-negative. Default is false.
%parallelism: @Is this just a python thing?

\end{itemize}



\newpage
\section{References}


\url{http://h2o.ai}

\url{http://github.com/0xdata/h2o.git}

\url{http://docs.0xdata.com}

\url{https://0xdata.atlassian.net/secure/Dashboard.jspa}

\url{https://groups.google.com/forum/#!forum/h2ostream}

"Strong Rules for Discarding Predictors in Lasso-type Problems" (\url{http://statweb.stanford.edu/~tibs/ftp/strong.pdf})

\end{document}
